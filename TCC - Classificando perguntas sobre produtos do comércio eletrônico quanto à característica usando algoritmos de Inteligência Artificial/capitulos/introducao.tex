\chapter[Introdução]{Introdução}
A invenção da \textit{World Wide Web} em 1989 mudou para sempre os hábitos das pessoas. A tecnologia cresceu rapidamente durante a década de 1990, e fontes como a pesquisa publicada por John Quarterman \cite{quarterman} e a União Internacional de Telecomunicações da ONU \cite{onu} relataram que o número de usuários da Internet cresceu de 2,62 milhões de pessoas em 1990 para 414 milhões de pessoas no ano 2000.

Esse cenário de rápido crescimento da Internet motivou empreendedores como Jeff Bezos a buscarem oportunidades nessa área. Inicialmente, ele criou um comércio totalmente online (ou seja, um \textit{e-commerce}) que vendia apenas livros, atuando como intermediário entre as editoras e os leitores \cite{loja_de_tudo}. Assim surgiu a \textit{Amazon}\footnote{\url{https://www.amazon.com}}. Na mesma época, concorrentes como o \textit{EBay}\footnote{\url{https://www.ebay.com/}}, plataforma que permitia que qualquer pessoa vendesse produtos (novos ou usados) num sistema de leilão \cite{manual_usuario}, também se destacavam nos Estados Unidos.

Após fazerem uma pós-graduação em administração de negócios na \textit{Stanford University}, nos Estados Unidos, dois argentinos resolveram se basear no modelo de sucesso do \textit{EBay} e criar um site de vendas de produtos através de leilões, levando a experiência de fazer compras em sites como o \textit{EBay} à Argentina e posteriormente ao Brasil \cite{manual_usuario}. Dessa forma, o Mercado Livre\footnote{\url{https://www.mercadolivre.com.br}} surgiu em 1999.

Quando os investidores dos Estados Unidos descobriram a Internet, eles a adotaram com devoção e uma bolha (chamada de Bolha PontoCom) começou a inflar. Os capitalistas de risco que investiam nos novos sites fornecedores de diferentes tipos de produtos, como o \textit{Pets.com}, de rações para animais, não necessariamente acreditavam que a Internet era a melhor maneira de vender comida para animais de estimação, mas sabiam que, se não financiassem essas empresas, seus concorrentes iriam financiá-las \cite{dot.con}. Essas novas empresas eram avaliadas em milhões de dólares a mais do que corporações tradicionais, como a Sears (loja de departamentos) e a Disney. Após maus resultados, muitas dessas novas empresas faliram.

Ao observar o cenário financeiro desfavorável do ano 2000, a \textit{Amazon} iniciou a modalidade de \textit{marketplace}, ou seja, passou a dispor de vários pequenos comerciantes em sua plataforma. Com isso, o seu catálogo de produtos tornou-se ainda maior, cativando o foco do cliente final no \textit{website} \cite{loja_de_tudo}. Percebendo a movimentação que a \textit{Amazon} fez para se proteger da crise, o Mercado Livre também decidiu se tornar um agregador de pequenas lojas \cite{manual_usuario}, se assemelhando ao seu concorrente estadunidense.

Com o passar do tempo, grandes lojas e os seus altos volumes de vendas passaram a fazer parte do Mercado Livre. A companhia passou a investir cada vez mais para buscar a liderança no setor e os esforços têm provocado resultado: no segundo trimestre de 2022, por exemplo, o volume bruto de mercadorias foi de US\$ 8,551 bilhões, o que representa um aumento de 21,7\% em relação ao mesmo período do ano anterior \cite{ml_report}. Ainda de acordo com o Relatório do Segundo Trimestre de 2022 do Mercado Livre, considerando apenas o Brasil, a receita em dólares aumentou 53\% ponderando os mesmos períodos \cite{ml_report}.

O rápido crescimento em número de clientes demonstrado pelas plataformas de \textit{e-commerce} trouxe consigo uma demanda cada vez maior por atendimento. Ao perguntar desde questões relacionadas à compatibilidade do produto anunciado com um outro item, até questões sobre o frete e tempo de entrega do produto, os clientes demonstram precisar de orientação para concluir a compra. De fato, uma pesquisa feita nos Estados Unidos concluiu que 99\% dos consumidores leem o campo de Perguntas e Respostas pelo menos ocasionalmente \cite{qna_survey}.

Um levantamento divulgado pelo grupo Ebit \cite{ebit} mostra que a quantia gasta em reais nos \textit{e-commerces} disponíveis no Brasil subiu 41\% entre os anos de 2019 e 2020, o que indica uma crescente adesão dos consumidores brasileiros ao \textit{e-commerce}. Essa grande demanda por atendimento preocupa os lojistas, que querem vender sem ter custos altos na resolução de perguntas. Surge uma solução: o uso de algoritmos e modelos de aprendizado de máquina, um campo de estudo em recente popularização. Líderes de novas empresas de tecnologia da informação perceberam que grande parcela das perguntas dos clientes de \textit{e-commerces} possuem respostas com métodos padronizados de se encontrar, pois já foram perguntadas por outras pessoas ou porque se referem a atributos daquele produto que já foram descritos pelo lojista ou pelo fabricante. Essas empresas de tecnologia da informação passaram então a fornecer serviços de inteligência artificial \cite{ai_in_ecommerce}, atendendo à nova demanda do mercado.

A depender das características da pergunta feita pelo cliente, um algoritmo diferente de aprendizado de máquina deve ser usado. Em \cite{df}, o modelo proposto responde a uma nova pergunta com a resposta dada por um atendente real a uma pergunta similar feita anteriormente. Ele usa aprendizado de máquina para classificar pares de perguntas como similares ou não similares, onde um elemento do par é uma nova pergunta ainda não respondida e o outro elemento do par é uma pergunta respondida existente na base de dados. No entanto, o método proposto em \cite{df} não é capaz de resolver o problema apresentado neste trabalho, pois para ser aplicado no problema seria preciso ter uma grande base de dados com pergunta e resposta sobre cada atributo de cada produto disponível no Mercado Livre, o que é custoso em termos financeiros e de trabalho humano. 

Em \cite{kg}, o modelo proposto responde a perguntas, usualmente sobre compatibilidade entre dois produtos, recuperando informações em um grafo de conhecimento. Ele usa aprendizado de máquina na etapa que consiste em aprimorar esse grafo de conhecimento, pois a recuperação de informações é feita a partir de pares de perguntas e respostas sobre compatibilidade escritas de forma não estruturada, e se faz necessário fornecer esses pares a um modelo de aprendizado de máquina para que sejam estruturados no formato adequado ao grafo de conhecimento \cite{kg}. Entretanto, a solução proposta não é capaz de responder as perguntas que este trabalho responde, pois ela consegue responder apenas questões relacionadas à compatibilidade entre dois produtos, e não questões relacionadas a outros atributos.

Este estudo propõe um novo método de resolução de perguntas de clientes e aplica-o em um cenário real. Para isso, técnicas da área de classificação de texto serão aplicadas em uma base de dados de perguntas sobre diferentes atributos de produtos do Mercado Livre. Após aplicar técnicas de processamento de linguagem natural e classificação de texto, será possível entender a qual atributo uma determinada pergunta se refere, e assim poder respondê-la recuperando o valor do atributo em uma tabela preenchida pelo lojista.

O trabalho desenvolvido foi uma solução realizada e aplicada na empresa GoBots como estágio do curso de graduação em Engenharia Mecatrônica.

\section{Objetivos}
\subsection{Objetivo geral}
O objetivo deste trabalho é propor um método de resolução de perguntas de clientes relacionadas a atributos específicos de um determinado produto para plataformas de \textit{e-commerce} atuantes no Brasil. Esse método é baseado em Classificação de Texto, mais especificamente Classificação de Perguntas, e será treinado a partir de uma base de dados de perguntas reais e aumentadas em Português.

Para chegar ao melhor modelo possível, foram avaliadas diferentes formas e ferramentas de pré-processamento de texto, de representação vetorial de texto (\textit{encoding}) e de classificação de texto. Esses componentes formarão uma \textit{pipeline} que classificará perguntas em atributos como ``cor'', ``marca'', ``tamanho'', entre outros. Ao final, os resultados desses experimentos em um conjunto de novas perguntas são discutidos.

\subsection{Objetivo específico}
\begin{itemize}
    \item Construção de uma base de dados rotulada em relação aos atributos de 1419 perguntas divididas em 28 ou 40 classes, sendo estas classes os atributos mais empregados na descrição dos produtos do Mercado Livre. 
    \item A base de dados é privada e pertence à empresa \textit{GoBots Soluções Inteligentes LTDA}.
    \item Teste de dois diferentes algoritmos de classificação de texto (BERT refinado em Português e \textit{DIETClassifier}), escolhendo o que se destaca mais com base em métricas conhecidas da literatura. A escolha dos algoritmos foi baseada em trabalhos anteriores da empresa GoBots, na qual este projeto foi desenvolvido.
    \item Estudo de caso da implementação da solução em um \textit{e-commerce} específico, o Mercado Livre, importante para avaliar se a solução seria útil também em outras empresas semelhantes brasileiras de \textit{e-commerce}. 
\end{itemize}

\section{Organização do Trabalho}
Os outros capítulos deste trabalho estão organizados da seguinte forma:
\begin{itemize}
    \item \textbf{Capítulo \ref{cap-revisao-bibliografica} --- Revisão Bibliográfica:} aprofunda a importância da resolução de perguntas dos clientes no atual contexto do \textit{e-commerce}, assim como explica os conceitos de Processamento de Linguagem Natural que serão usados no decorrer do trabalho. O capítulo também apresenta trabalhos relacionados considerando a aplicação ou a língua dos dados nos quais foram treinados.
    \item \textbf{Capítulo \ref{cap-desenvolvimento} --- Método para Avaliação de Classificadores Treinados na Base de Dados do Mercado Livre:} descreve cada um dos passos percorridos para treinar os diferentes modelos de Classificação de Texto apresentados, desde a definição das classes e coleta de dados até o uso da plataforma de treinamento em si, bem como apresenta a forma como foi feita a avaliação dos modelos.
    \item \textbf{Capítulo \ref{cap-resultados} --- Resultados:} compara as métricas encontradas para os modelos experimentados em diferentes situações, analisando pontos positivos e negativos de acordo com a capacidade deles em classificar perguntas quanto a atributos.
    \item \textbf{Capítulo \ref{cap-conclusao} --- Conclusão:} resume os resultados encontrados e as análises feitas, assim como levanta pontos a serem aprimorados em trabalhos futuros.
\end{itemize}